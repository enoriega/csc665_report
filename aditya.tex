\section{WordNet Graph Traversal}
Similar to most ideas that implement soft inference, we approached the task of chaining
question/answer word pairs directly by generating meaningful paths of logical consequence
with the use of WordNet [\ref{wordnet}] graph traversals between entities formed from
each question and answer. Since the mere existence of these paths alone cannot legitimately
explain the mapping from a question to an answer, we use the word embeddings of the dataset
trained over a scientific knowledge base. The word embeddings are generated with the 
continuous bag-of-words (CBOW) model, or \emph{word2vec}, developed by Mikolov et al. 2013 [\ref{w2v}]
The vectors are trained from a concatenated text of \emph{Campbell's Dictionary, Barron's Flash Cards,
and Simple English Wikipedia}.

\subsection{Construction of graphs}
At the crux of it, WordNet is already a graph of word senses, called synsets. Each synset has 
an associated part-of-speech (POS) tag associated with it. In the context of QA classification,
each word in the question and answer ideally constitute to the start vertex of the graph.
To reduce noise, we only consider nouns, verbs, and adjectives extracted from annotating the 
questions and answers with the help of the \emph{processors} package [\ref{processors}]. 
In addition, various stop words are filtered from the set of tokens to retain only the focus words. 
The resultant set becomes the set of start vertices $S$ from which edge information is extracted.

\subsubsection{Word sense disambiguation}
Given a word \emph{w} in isolation, there are multiple selection criteria to pick and choose the 
best word sense for a particular context - the simplest being picking the top synset returned by
WordNet. To direct the search toward the context of the question/answer, we use a simple similarity
function between the current synset and the context. We define the vector representation of a 
synset to be the average of the vectors of the lemmas that represent the synset, and the vectors of
the tokens in the glossary/definition of the synset. This method simply attempts to encompass the
context of the synset against the knowledge base in use. The best synset for the word, \emph{w} then
becomes the synset with the highest \emph{cosine similarity} between the context - represented
by the average vector of the set of tokens in the question/answer, and vector of the synset.
Formally, the top-$k$ synsets associated with a word, \emph{w} is given by
$top-K(s, \overline{(\vec{s_{def}}, \vec{s_{set}})}\ \textbf{.}\ \overline{ref})$ where $ref$ is
the context defined by question/answer.

\subsubsection{Selecting edges}
For every POS, there are associated word relations that constitute the WordNet graph itself. We 
only consider the following word associations: 

\begin{enumerate}
\item Hypernymy/hyponymy chains that represent the \emph{is-a} relation.
\item Holonymy/meronymy chains that represent the \emph{part-of} and \emph{belongs-to} relations.
\item Similar words that constitute the synonyms of the current synset.
\end{enumerate}

Given a start vertex $s$, we simply perform a Breadth-First Search (BFS) adding the neighbours of
the current node (prescribed by the above edge criteria) to a queue, and iteratively adding the
neighbours of the nodes in queue. It must be noted that we only traverse the holonymy/hypernymy
chain and manually add a corresponding meronym/hyponym edge since this greatly reduces the search
space. Adding the hyponyms/meronyms of the root node not only adds great semantic drift, but 
also causes the graph order to explode. We use the well-defined implementation of graphs by
scala-graph [\ref{scala-graph}] to run the tests.

\subsection{Listing paths}
Given two nodes \emph{v} and \emph{t}, a simple Depth-First Search (DFS) exhaustively lists
all paths from \emph{v} to \emph{t} as observed below:

\begin{algorithm}
\caption{Generate all paths}
  \begin{algorithmic}[1]
    \Procedure{GENERATE-ALL-PATHS}{G, visited, paths, curPath, v, t}
    \State $curPath.push(v)$
    \State $visited\ +=\ v$
    \If{$v\ ==\ t$}
      \State $paths\ +=\ curPath$
      \Else 
        \State $w \gets v.neighbors$
          \If{$!(visited.contains(w))$}
            \State GENERATE-ALL-PATHS(G, visited, paths, curPath, w, t)
          \EndIf
    \EndIf
    \State $curPath.pop$
    \State $visited\ -=\ v$
    \EndProcedure
  \end{algorithmic}
\end{algorithm}

\subsection{Scoring a path}
Given a question \emph{q} and an associated answer \emph{a}, we define the score of a path \emph{p},
as $$S(q, a, p) = \left(\sum_{k}\log\left\{((\vec{q}+\vec{a}).(\vec{p_{def}}+\vec{p_{set}}))\times N(p_k)\right\}\right)$$\\
where $N(p_k)$ represents a penalty function based on the position of node in the hypernymy/holonymy tree.
This is done to ensure that nodes closer to the leaves of the tree are scored higher are than the ones
near the root(s). $N(p_k)$ is simply 1/$h'$ where $h'$ is the normalized depth of the node in the hypernym tree.
We sum the logs fundamentally to ensure that longer paths do not get more weightage simply by their
virtue of longer length.

\subsection{Features}
Clearly, there are more than one path from an entity in the question to an entity in the answer,
more than one entity per question/answer, and more than one answer per question. To essentially
train a machine learning model that classifies simply based on out scoring function, we define
multiple features that will dictate how the classifier makes a predictions on. 

\begin{enumerate}
\item \# of paths, the average score, and the average path length.
\item The min/max score, and their corresponding path lengths.
\item The min/max path lengths, and their corresponding scores.
\item Graph order: node size, edge size.
\item Ratio of hypernyms to hyponyms and holonyms to meronyms.
\item \emph{word2vec} score between question and answer.
\end{enumerate}

For each question, answer pair, we also look at paths between entities in the question/answer itself.
We do this because some questions do not have a recognizable entity in the answer and the scores
invariably become a vector of zeroes. We add a ``\emph{qa\_}'' prefix to differentiate these
features from the paths denoting \emph{q} $\rightarrow$ \emph{a}.

\subsection{Results}
After training 1994 questions from the kaggle AI2 datatset, we tested over a 506 question dataset
as development set. After tuning the hperparameters after each run, table \ref{wngraph} summarizes
the results we obtained with only the graph traversal as the classifier. It is important to note
that training on the aritificial questions did not increase the performance by a great factor,
because each graph is built conjunctively with entities from both the question and the answer.
Since the artificial questions essentially contain the same focus words, the scores remain virtually
the same.

\begin{table}[htp]
\caption{Accuracy of the system}
\begin{center}
\begin{tabular}{|c|c|c|c|c|c|}
\hline
C&0.1&1.0&10.0&100.0&1000.0\\\hline
Accuracy&0.247&0.274&0.25&0.274&0.241\\
\hline
\end{tabular}
\end{center}
\label{wngraph}
\end{table}

Clearly, the system is not very informative by itself, performing barely better than chance. Inspecting
the scores of individual answers, it was evident that the scores were very similar between each answer
which emphasizes that each question/answer word pair is very tightly packed with comparable graph
orders. Therefore, it is prudent to use a knowledge graph sculpted solely for QA classification in the
science domain, since the word2vec scores of these will be vastly different compared to a dictionary
like WordNet.