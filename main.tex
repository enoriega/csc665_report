%
% File acl2016.tex
%
%% Based on the style files for ACL-2015, with some improvements
%%  taken from the NAACL-2016 style
%% Based on the style files for ACL-2014, which were, in turn,
%% Based on the style files for ACL-2013, which were, in turn,
%% Based on the style files for ACL-2012, which were, in turn,
%% based on the style files for ACL-2011, which were, in turn, 
%% based on the style files for ACL-2010, which were, in turn, 
%% based on the style files for ACL-IJCNLP-2009, which were, in turn,
%% based on the style files for EACL-2009 and IJCNLP-2008...

%% Based on the style files for EACL 2006 by 
%%e.agirre@ehu.es or Sergi.Balari@uab.es
%% and that of ACL 08 by Joakim Nivre and Noah Smith

\documentclass[11pt]{article}
\usepackage{acl2016}
\usepackage{times}
\usepackage{url}
\usepackage{latexsym}
\usepackage{hyperref}
\usepackage{algorithm}
\usepackage{algpseudocode}

\aclfinalcopy % Uncomment this line for the final submission
%\def\aclpaperid{***} %  Enter the acl Paper ID here

%\setlength\titlebox{5cm}
% You can expand the titlebox if you need extra space
% to show all the authors. Please do not make the titlebox
% smaller than 5cm (the original size); we will check this
% in the camera-ready version and ask you to change it back.

\newcommand\BibTeX{B{\sc ib}\TeX}

\title{Csc 655 Final Report}

\author{Rebecca Sharp, Aditya Kousik, Enrique Noriega, Benjamin Martin\\
  {\tt \{bsharp,adityak,enoriega,bamartin\}@email.arizona.edu} \\}
 
\date{}

%\author{Enrique Noriega \\
%{\tt enoriega@email.arizona.edu} \\\And
%Rebecca Sharp \\
%{\tt bsharp@email.arizona.edu} \\\And
%Aditya Kousik \\
%{\tt adityak@email.arizona.edu} \\\And
%Benjamin Martin \\
%{\tt bamartin@email.arizona.edu} 
%}

\begin{document}
\maketitle
\begin{abstract}
  
\end{abstract}

\section{Introduction}

Question Answering (QA) is a challenging problem. Systems attempting to solve the problem still don't achieve performance comparable to a human. there are multiple reasons for this. One of them is that many questions require \emph{soft inference}, where traditional information retrieval methods fail. Another reason is that training data is scarce, and coming up with more training data is expensive and requires a lot of effort. \\

During the course of Spring 2016, we pursued some ideas to tackle these two issues. At the end we compare how well the solutions work individually as well as combined in an ensemble model, demonstrating the degree of complementarity. The code implementing our ideas can be found at:\\ 
\url{https://github.com/enoriega/cs665.git}\\

To do evaluation of our system and our ideas, we use the \emph{AI2 dataset} from its Kaggle challenge.
\section{Information Retrival}
Our starting point was to develop an \emph{Information Retrieval} system. Traditionally these have been used for automatic question answering. The principle behind them is simple: Query a knowledge base with a question and a candidate answer and fetch a results set.  Gather the result sets of all the candidate answers to a questions and rank them. The highest ranked answer set will be the choice of the system.\\

Our QA system uses Apache Lucene [\ref{lucene}] as its IR engine. The knowledge base is the \emph{Simple English Wikipedia \(Simple Wiki\)}. We indexed Simple Wiki using the default Standard Analyzer, which tokenizes white by white spaces and removes english stop words. To rank the question-answer vectors we use \emph{svm\_rank} [\ref{svmrank}].

\subsection{Features}

We use the following features to train and test the ranker:

\begin{enumerate}
\item Top \emph{tf-idf} score of the query
\item \# of query hits in the index
\item Sum of the \emph{top 100} tf-idf scores divided by the number of documents in the index
\item \# of items from the \emph{top 100} hits with a normalized tdf-idf score in the bins $[0, 0.25], [0.26, 0.50], [0.51, 0.75],  [0.76, 1.00]$
\end{enumerate}

The motivation is to rank each QA pair by measuring how much evidence supports the candidate answer in the knowledge base (features 1 and 2) as well as how \emph{confident} is the evidence (features 3 and 4). Observe that feature 4 are really four features, the quartile bins.\\

\subsection{Results}


\section{Data Augmentation}

One drawback of QA is the lack of quality training data. Annotating a corpus to extract multiple choice questions is an expensive endeavor, for example, AI2, with a high amount of funding available only provided a training data set of 2.5K multiple choice questions.\\
On the other hand. There is a lot of information available in the form of other resources, like textbooks, where it is encoded as sentences that aren't necessarily a question. We use this observation to devise a technique to extract candidate questions from these sentences and expand the size of our training data set with the hope to increse the performance of machine learning strategies for QA.\\

\subsection{One pattern - Noun Phrases}

We use the \emph{CoreNLPProcessor}[\ref{corenlp}] class from \emph{Processors}[\ref{processors}] to do a dependency parse over a set of sentences. For each sentence, if there is a noun phrase linked to a root verb, we consider the subtree spanning from an \emph{nsubj} dependency as an answer, and the rest of the tree as a question. Algorithm \ref{nounphrase} depicts the algorithm in the form of pseudocode.\\

\begin{algorithm}
\caption{NPhrase QA extraction}\label{nounphrase}
\begin{algorithmic}[1]
\Procedure{qa\_extraction}{$sen$}
   \State $deps = dep\_parse(sen)$
   \State $root = get_root(deps)$
   \State $out\_edges = get_out_edges(root)$
   \ForAll {$edge \in out\_edges$}
   	\If {$edge.label == nsubj$}
		\State $a = spanning\_tree(edge)$
		\State $q = deps - a$
		\State $qa = (words(q), words(a))$
		\State \textbf{return} $qa$
	\EndIf
   \EndFor
   \EndProcedure
\end{algorithmic}
\end{algorithm}
   
\subsection{Restrictions}
We can't just use any sentence as the source of an \emph{artificial qa}, as many pairs would have an informative question and/or answer. Applying some heuristics allow us to harvest informative mostly qa pairs from text books. The heuristic are applied as filters in cascade:
\begin{itemize}
\item \textbf{Stop words}:Ignore sentences the with the following words in their noun phrase: ``figure'', ``table'', ``example'', ``chapter''
\item \textbf{POS Tags}: Ignore those that only have one word, whose POS tag is either ``PRP'' or ``DT''
\item \textbf{Nouns}: Consider only those that have at least one noun (A word that has a POS tag starting with N)
\item \textbf{Nouns again}: Consider only those sentences that have at least a noun in their verbal phrase
\end{itemize}

\subsection{Alternative answers}

Since we are talking about multiple choice answers, we need three alternative answers for each collected qa pair. To generate these, let's define the \emph{neighborhood} of a question as the $k$ sentences before and the $k$ sentences after in the source document, excluding the considered qa pair itself. Any answer from any question in the neighborhood of $qa$ is an alternative answer for it as long as the following assertions hold:

\begin{itemize}
\item \textbf{No reduncance}: It is not the same answer as the correct answer
\item \textbf{Substance}: There is at least a different noun than in the correct answer
\end{itemize}

\subsection{Statistics}

We ran the algorithm over several textbooks and knowledge bases to collect artificial questions. Table \ref{artificialqa} summarizes the amount of collected questions from each resource.

\begin{table}[htp]
\caption{Amount of collected artificial \emph{QA}s}
\begin{center}
\begin{tabular}{|c|c|c|}
\hline
Resource&\# of sen.& \# of artificial QAs\\
\hline
CK 12 biology & 16000 & 5000\\
Simple Wiki & 1 & 1 \\
\hline
\end{tabular}
\end{center}
\label{artificialqa}
\end{table}%

We can see that the amount of artificial questions exceed the amount of training data provided for the Kaggle challenge.

\subsection{IR + Artificial QAs}
We train the IR system using the original training data set and the artificial questions to see if there is a difference in the performance.


\section{WordNet Graph Traversal (Aditya)}
Similar to most ideas that implement soft inference, we approached the task of chaining
question/answer word pairs directly by generating meaningful paths of logical consequence
with the use of WordNet [\cite{wordnet}] graph traversals between entities formed from
each question and answer. Since the mere existence of these paths alone cannot legitimately
explain the mapping from a question to an answer, we use the word embeddings of the dataset
trained over a scientific knowledge base. The word embeddings are generated with the 
continuous bag-of-words (CBOW) model, or \emph{word2vec}, developed by Mikolov et al. 2013 [\cite{w2v}]
The vectors are trained from a concatenated text of \emph{Campbell's Dictionary, Barron's Flash Cards,
and Simple English Wikipedia}.

\subsection{Construction of graphs}
At the crux of it, WordNet is already a graph of word senses, called synsets. Each synset has 
an associated part-of-speech (POS) tag associated with it. In the context of QA classification,
each word in the question and answer ideally constitute to the start vertex of the graph.
To reduce noise, we only consider nouns, verbs, and adjectives extracted from annotating the 
questions and answers with the help of the \emph{processors} package [\cite{processors}].
In addition, various stop words are filtered from the set of tokens to retain only the focus words. 
The resultant set becomes the set of start vertices $S$ from which edge information is extracted.

\subsubsection{Word sense disambiguation}
Given a word \emph{w} in isolation, there are multiple selection criteria to pick and choose the 
best word sense for a particular context - the simplest being picking the top synset returned by
WordNet. To direct the search toward the context of the question/answer, we use a simple similarity
function between the current synset and the context. We define the vector representation of a 
synset to be the average of the vectors of the lemmas that represent the synset, and the vectors of
the tokens in the glossary/definition of the synset. This method simply attempts to encompass the
context of the synset against the knowledge base in use. The best synset for the word, \emph{w} then
becomes the synset with the highest \emph{cosine similarity} between the context - represented
by the average vector of the set of tokens in the question/answer, and vector of the synset.
Formally, the top-$k$ synsets associated with a word, \emph{w} is given by
$top-K(s, \overline{(\vec{s_{def}}, \vec{s_{set}})}\ \textbf{.}\ \overline{ref})$ where $ref$ is
the context defined by question/answer.

\subsubsection{Selecting edges}
For every POS, there are associated word relations that constitute the WordNet graph itself. We 
only consider the following word associations: 

\begin{enumerate}
\item Hypernymy/hyponymy chains that represent the \emph{is-a} relation.
\item Holonymy/meronymy chains that represent the \emph{part-of} and \emph{belongs-to} relations.
\item Similar words that constitute the synonyms of the current synset.
\end{enumerate}

We take a start vertex $s$ and simply perform a Breadth-First Search (BFS) adding the neighbours of
the current node (prescribed by the above edge criteria) to a queue, and iteratively adding the
neighbours of the nodes in queue. It must be noted that we only traverse the holonymy/hypernymy
chain and manually add a corresponding meronym/hyponym edge since this greatly reduces the search
space. Adding the hyponyms/meronyms of the root node not only adds great semantic drift, but 
also causes the graph order to explode. We use the well-defined implementation of graphs by
scala-graph [\cite{scala-graph}] to run the tests.

\subsection{Listing paths}
For any two nodes \emph{v} and \emph{t}, a Depth-First Search (DFS) exhaustively lists
all paths from \emph{v} to \emph{t} as observed below:

\begin{algorithm}
\caption{Generate all paths}
  \begin{algorithmic}[1]
    \Procedure{GENERATE-ALL-PATHS}{G, visited, paths, curPath, v, t}
    \State $curPath.push(v)$
    \State $visited\ +=\ v$
    \If{$v\ ==\ t$}
      \State $paths\ +=\ curPath$
      \Else 
        \State $w \gets v.neighbors$
          \If{$!(visited.contains(w))$}
            \State GENERATE-ALL-PATHS(G, visited, paths, curPath, w, t)
          \EndIf
    \EndIf
    \State $curPath.pop$
    \State $visited\ -=\ v$
    \EndProcedure
  \end{algorithmic}
\end{algorithm}

\subsection{Scoring a path}
We define the score of a path \emph{p} for a question \emph{q} and an associated answer \emph{a},
as $$S(q, a, p) = \left(\sum_{k}\log\left\{((\vec{q}+\vec{a}).(\vec{p_{def}}+\vec{p_{set}}))\times N(p_k)\right\}\right)$$\\
where $N(p_k)$ represents a penalty function based on the position of node in the hypernymy/holonymy tree.
This is done to ensure that nodes closer to the leaves of the tree are scored higher are than the ones
near the root(s). $N(p_k)$ is simply 1/$h'$ where $h'$ is the normalized depth of the node in the hypernym tree.
We sum the logs fundamentally to ensure that longer paths do not get more weightage simply by their
virtue of longer length. The process is very CPU-intensive when run over the set of all questions $Q$. Therefore,
we parallelised the training and loading of graphs to speedup the execution.


\subsection{Features}
Clearly, there are more than one path from an entity in the question to an entity in the answer,
more than one entity per question/answer, and more than one answer per question. To essentially
train a machine learning model that classifies simply based on out scoring function, we define
multiple features that will dictate how the classifier makes a predictions on. 

\begin{enumerate}
\item \# of paths, the average score, and the average path length.
\item The min/max score, and their corresponding path lengths.
\item The min/max path lengths, and their corresponding scores.
\item Graph order: node size, edge size.
\item Ratio of hypernyms to hyponyms and holonyms to meronyms.
\item \emph{word2vec} score between question and answer.
\end{enumerate}

For each question, answer pair, we also look at paths between entities in the question/answer itself.
We do this because some questions do not have a recognizable entity in the answer and the scores
invariably become a vector of zeroes. We add a ``\emph{qa\_}'' prefix to differentiate these
features from the paths denoting \emph{q} $\rightarrow$ \emph{a}.

\subsection{Results}
After training 1994 questions from the kaggle AI2 datatset, we tested over a 506 question dataset
as development set. After tuning the hperparameters after each run, table \ref{wngraph} summarizes
the results we obtained with only the graph traversal as the classifier. It is important to note
that training on the aritificial questions did not increase the performance by a great factor,
because each graph is built conjunctively with entities from both the question and the answer.
Since the artificial questions essentially contain the same focus words, the scores remain virtually
the same.

\begin{table}[htp]
\caption{Accuracy of the system}
\begin{center}
\begin{tabular}{|c|c|c|c|c|c|}
\hline
C&0.1&1.0&10.0&100.0&1000.0\\\hline
Accuracy&0.247&0.274&0.25&0.274&0.241\\
\hline
\end{tabular}
\end{center}
\label{wngraph}
\end{table}

It is evident the system is not very informative by itself, performing barely better than chance. Inspecting
the scores of individual answers, it was observed that the scores were very similar between each answer
which emphasizes that each question/answer word pair is very tightly packed with comparable graph
orders. Therefore, it is prudent to use a knowledge graph sculpted solely for QA classification in the
science domain, since the word2vec scores of these will be vastly different compared to a dictionary
like WordNet.

\section{Relation Vectors (Ben)}

	One of the ways that one could tell that the answer to \textit{Which example describes a learned behavior in a dog?} is \textit{Sitting on command} is that the relationship between \textit{dog} and \textit{command} is similar the meaning of the word \textit{learn}. Importantly, the meaning is not \emph{exactly} the same as \textit{learn}, only similar. In order to attempt to capture this kind of fact, three kinds of embeddings were learned. An embedding matrix $V$ for two-place predicates was learned, such that, for example, $\mathbf{v}_{learn}$ would be a vector in a 200-dimensional real vector space. Embedding matrices $S$ and $O$, for arguments to two-place predicates, were also learned, so $\mathbf{s}_{dog}$ and $\mathbf{o}_{command}$ were also vectors in 200-dimensional real vector spaces. Finally, a $200\times 400$ weight matrix $W$ was also learned, such that if the relationship between $i$ and $j$ is approximately the same as the meaning of $k$, then $tanh(W \langle \mathbf{s}_{i} , \mathbf{o}_{j} \rangle) \approx \mathbf{v}_{k}$. 
    
    This was all implemented as a Neural Network, using Lasagne\footnote{\url{github.com/Lasagne/Lasagne}}, which is built on Theano[\ref{THEANO2,THEANO1}]. Data was taken from Simple English Wikipedia, parsed using Processors[\ref{processors}]. After excluding a manually-defined list of dependency types, roots which took exactly two arguments were selected, and used to train the network using a Mean Squared Error cost function and Nesterov momentum.
    
   	The results were not good. While the training generalized to a held-out validation data set consisting of $\%10$ of the Simple Wiki data, this did not translate to sensible embeddings. For example, the cosine similarity between the vector predicted for the noun pair \textit{dog}+\textit{command} and the embedding learned for \textit{learn} was $-.09$, meaning that they were essentially unrelated, and even slightly pointed in opposite directions. Using these vectors to answer questions for the challenge was no better than chance ($\%25.8$), so these vectors were not included in the overall Ensemble model.

\section{Ensemble (Becky)}
\label{sec:ensemble}

We combined each of our models into a voting ensemble.  Rather than each model casting either a single vote for one answer candidate only, each was able to cast 100 votes, distributed across all answer candidates according to the individual model's precision at each rank.

To determine the precision at rank (P@R) for a given model, we ran the model over the development data~\footnote{The development data used here were the 506 questions.}.  For each question, the model ranked the four candidate answers.  The P@R for rank 0 was determined by looking at all of the answer candidates which were ranked in the top position and calculating the precision of those selections.  This is identical to determining the precision at 1 for the model.  Then, to find the P@R 1, we gathered, instead, all of the candidate answers which we ranked in second place and calculated their precision.  The same was done for ranks 2 and 3.  The P@R for a given rank determined how many of the votes were cast by the model to each of the answer candidates in that rank.  The P@Rs for each model in the ensemble are shown in Table \ref{tab:p@r}.

\begin{table}[H]
\caption{Precision at rank (as a percentage correct at each rank) for each of the models, as determined over the 506 development questions.  These were used to allow for voting scales customized to each model.}
\begin{center}
\begin{tabular}{lcccc}
%\hline
Model & P@R0 & P@R1 & P@R2 & P@R3\\
\hline
IR\_SimpleEng & 34.8 & 26.1 & 20.9 & 18.2 \\
Paths &  &  &  &  \\
Ben's &  &  &  &  \\

\end{tabular}
\end{center}
\label{tab:p@r}
\end{table}%

\begin{table}[H]
\caption{Performance (given as precision at 1, P@1) of the ensemble when using the personalized voting scales, a set 4-3-2-1 voting scale (Standard), and when allowing models to cast only one vote for their top-ranked candidate (Single).}
\begin{center}
\begin{tabular}{lc}
%\hline
Voting Method & P@1 \\
\hline
Single 		& 53.9\%	\\
Standard 	& 52.9\% \\
Personalized 	& 56.5\% 	\\

\end{tabular}
\end{center}
\label{tab:ensemblemethods}
\end{table}%	
	
To determine the utility of the personalized voting scale, we compared the impact of voting method on the performance of the ensemble.  For evaluation we used the 506 development questions, and the models which were included were: our IR over simple English wikipedia, the IR of team CS, the best-performing soft-inference feature of team CS, and the system of team MIS.  The voting methods we compared were the personalized voting scale, a standardized voting scale (i.e. 4 votes for rank 0, 3 votes for rank 1, 2 votes for rank 2, and 1 vote for rank 3 for all models), as well as a single vote for only the highest ranked answer candidate.  The comparison of the results of these voting methods is provided in Table~\ref{tab:ensemblemethods}.  By allowing each model to have a personalized voting scale, we had the highest performance (56.5\% P@1, a relative gain of about 5\% over the single-vote method and about 7\% relative gain over the standard-scale method).

\section{Conclusions}

%\bibliography{acl2016}
%\bibliographystyle{acl2016}

\end{document}
