\section{Ensemble (Becky)}
\label{sec:ensemble}

We combined each of our models into a voting ensemble.  Rather than each model casting either a single vote for one answer candidate only, each was able to cast 100 votes, distributed across all answer candidates according to the individual model's precision at each rank.

To determine the precision at rank (P@R) for a given model, we ran the model over the development data~\footnote{The development data used here were the 506 questions.}.  For each question, the model ranked the four candidate answers.  The P@R for rank 0 was determined by looking at all of the answer candidates which were ranked in the top position and calculating the precision of those selections.  This is identical to determining the precision at 1 for the model.  Then, to find the P@R 1, we gathered, instead, all of the candidate answers which we ranked in second place and calculated their precision.  The same was done for ranks 2 and 3.  The P@R for a given rank determined how many of the votes were cast by the model to each of the answer candidates in that rank.  The P@Rs for each model in the ensemble are shown in Table \ref{tab:p@r}.

\begin{table}[H]
\caption{Precision at rank (as a percentage correct at each rank) for each of the models, as determined over the 506 development questions.  These were used to allow for voting scales customized to each model.}
\begin{center}
\begin{tabular}{lcccc}
%\hline
Model & P@R0 & P@R1 & P@R2 & P@R3\\
\hline
IR\_SimpleEng & 34.8 & 26.1 & 20.9 & 18.2 \\
Paths &  &  &  &  \\
Ben's &  &  &  &  \\

\end{tabular}
\end{center}
\label{tab:p@r}
\end{table}%

\begin{table}[H]
\caption{Performance (given as precision at 1, P@1) of the ensemble when using the customized voting scales, a set 4-3-2-1 voting scale (Standard), and when allowing models to cast only one vote for their top-ranked candidate (Single).}
\begin{center}
\begin{tabular}{lc}
%\hline
Voting Method & P@1 \\
\hline
Single 		& 	\\
Standard 	&   \\
Customized 	&  	\\

\end{tabular}
\end{center}
\label{tab:ensemblemethods}
\end{table}%	
	
We compared the performance of the ensemble with this personalized voting to that of a model which used both a set voting scale (i.e. 4 votes for rank 0, 3 votes for rank 1, 2 votes for rank 2, and 1 vote for rank 3) as well as a single vote for only the highest ranked answer candidate.  The comparison of the results of these voting methods is provided in Table~\ref{tab:ensemblemethods}.