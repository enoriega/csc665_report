\section{Information Retrival}
Our starting point was to develop an \emph{Information Retrieval} system. Traditionally these have been used for automatic question answering. The principle behind them is simple: Query a knowledge base with a question and a candidate answer and fetch a results set.  Gather the result sets of all the candidate answers to a questions and rank them. The highest ranked answer set will be the choice of the system.\\

Our QA system uses Apache Lucene [\ref{lucene}] as its IR engine. The knowledge base is the \emph{Simple English Wikipedia \(Simple Wiki\)}. We indexed Simple Wiki using the default Standard Analyzer, which tokenizes white by white spaces and removes english stop words. To rank the question-answer vectors we use \emph{svm\_rank} [\ref{svmrank}].

\subsection{Features}

We use the following features to train and test the ranker:

\begin{enumerate}
\item Top \emph{tf-idf} score of the query
\item \# of query hits in the index
\item Sum of the \emph{top 100} tf-idf scores divided by the number of documents in the index
\item \# of items from the \emph{top 100} hits with a normalized tdf-idf score in the bins $[0, 0.25], [0.26, 0.50], [0.51, 0.75],  [0.76, 1.00]$
\end{enumerate}

The motivation is to rank each QA pair by measuring how much evidence supports the candidate answer in the knowledge base (features 1 and 2) as well as how \emph{confident} is the evidence (features 3 and 4). Observe that feature 4 are really four features, the quartile bins.\\

\subsection{Results}


\section{Data Augmentation}

One drawback of QA is the lack of quality training data. Annotating a corpus to extract multiple choice questions is an expensive endeavor, for example, AI2, with a high amount of funding available only provided a training data set of 2.5K multiple choice questions.\\
On the other hand. There is a lot of information available in the form of other resources, like textbooks, where it is encoded as sentences that aren't necessarily a question. We use this observation to devise a technique to extract candidate questions from these sentences and expand the size of our training data set with the hope to increse the performance of machine learning strategies for QA.\\

\subsection{One pattern - Noun Phrases}

We use the \emph{CoreNLPProcessor}[\ref{corenlp}] class from \emph{Processors}[\ref{processors}] to do a dependency parse over a set of sentences. For each sentence, if there is a noun phrase linked to a root verb, we consider the subtree spanning from an \emph{nsubj} dependency as an answer, and the rest of the tree as a question. Algorithm \ref{nounphrase} depicts the algorithm in the form of pseudocode.\\

\begin{algorithm}
\caption{NPhrase QA extraction}\label{nounphrase}
\begin{algorithmic}[1]
\Procedure{qa\_extraction}{$sen$}
   \State $deps = dep\_parse(sen)$
   \State $root = get_root(deps)$
   \State $out\_edges = get_out_edges(root)$
   \ForAll {$edge \in out\_edges$}
   	\If {$edge.label == nsubj$}
		\State $a = spanning\_tree(edge)$
		\State $q = deps - a$
		\State $qa = (words(q), words(a))$
		\State \textbf{return} $qa$
	\EndIf
   \EndFor
   \EndProcedure
\end{algorithmic}
\end{algorithm}
   
\subsection{Restrictions}
We can't just use any sentence as the source of an \emph{artificial qa}, as many pairs would have an informative question and/or answer. Applying some heuristics allow us to harvest informative mostly qa pairs from text books. The heuristic are applied as filters in cascade:
\begin{itemize}
\item \textbf{Stop words}:Ignore sentences the with the following words in their noun phrase: ``figure'', ``table'', ``example'', ``chapter''
\item \textbf{POS Tags}: Ignore those that only have one word, whose POS tag is either ``PRP'' or ``DT''
\item \textbf{Nouns}: Consider only those that have at least one noun (A word that has a POS tag starting with N)
\item \textbf{Nouns again}: Consider only those sentences that have at least a noun in their verbal phrase
\end{itemize}

\subsection{Alternative answers}

Since we are talking about multiple choice answers, we need three alternative answers for each collected qa pair. To generate these, let's define the \emph{neighborhood} of a question as the $k$ sentences before and the $k$ sentences after in the source document, excluding the considered qa pair itself. Any answer from any question in the neighborhood of $qa$ is an alternative answer for it as long as the following assertions hold:

\begin{itemize}
\item \textbf{No reduncance}: It is not the same answer as the correct answer
\item \textbf{Substance}: There is at least a different noun than in the correct answer
\end{itemize}

\subsection{Statistics}

We ran the algorithm over several textbooks and knowledge bases to collect artificial questions. Table \ref{artificialqa} summarizes the amount of collected questions from each resource.

\begin{table}[htp]
\caption{Amount of collected artificial \emph{QA}s}
\begin{center}
\begin{tabular}{|c|c|c|}
\hline
Resource&\# of sen.& \# of artificial QAs\\
\hline
CK 12 biology & 16000 & 5000\\
Simple Wiki & 1 & 1 \\
\hline
\end{tabular}
\end{center}
\label{artificialqa}
\end{table}%

We can see that the amount of artificial questions exceed the amount of training data provided for the Kaggle challenge.

\subsection{IR + Artificial QAs}
We train the IR system using the original training data set and the artificial questions to see if there is a difference in the performance.
